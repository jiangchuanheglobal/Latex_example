%%%%%%%%%%%%%%%%%%%%%%%%%%%%%%%%%%%%%%%%%%%%%%%%%%%%%%%%%%%%%%%%%%
%%%%%%%% ICML 2013 EXAMPLE LATEX SUBMISSION FILE %%%%%%%%%%%%%%%%%
%%%%%%%%%%%%%%%%%%%%%%%%%%%%%%%%%%%%%%%%%%%%%%%%%%%%%%%%%%%%%%%%%%

% Use the following line _only_ if you're still using LaTeX 2.09.  %\documentstyle[icml2013,epsf,natbib]{article}
% If you rely on Latex2e packages, like most moden people use this:
\documentclass{article}

% For figures
\usepackage{graphicx} % more modern
%\usepackage{epsfig} % less modern
\usepackage{subfigure} 

% For citations
\usepackage{natbib}

% For algorithms
\usepackage{algorithm}
\usepackage{algorithmic}

\usepackage{hyperref}

\newcommand{\theHalgorithm}{\arabic{algorithm}}

\usepackage[accepted]{icml2013} 

\icmltitlerunning{CSE674 Advanced Machine Learning Project}

\begin{document} 

\twocolumn[
\icmltitle{Hande Writing Probability Inference and Classification \\ 
		}

\icmlauthor{Jiangchuan He}{jiangchu@buffalo.edu}
\icmladdress{Computer Science and Engineering Department, University at Buffalo,
            Amherst, Buffalo, NY, USA}

% the "keywords" metadata in the PDF but will not be shown in the document
\icmlkeywords{boring formatting information, machine learning, ICML}

\vskip 0.3in
]

\begin{abstract} 
As we know, Handwriting recogition is one of the hot area in Machine Learning and
Pattern Classification. Using probabilist graphical model on handwriting
data is a commom way to achieve inference and classification.
In this article, I am going to use probabilist graphical model (PGM) to make
inference on children handwriting data set. I will also give specific implementation
as well as evaluation of algorithms.
\end{abstract} 

\section{Introduction}
In the past few decades, Machine Learning has increasingly gain its reputation in realizing
artificial intellgience. A huge amount of machine learning algorithms have been invented to
achieve more and more precise inference and prediction goal. This artical mainly concerned
with one of them called probabilistic graphic model. The advantage of this model is that it
can help infer the random variables dependence and thus may have better result for our 
handewriting analysis.
\subsection{General Description}
In this project, I will use one of the model, call probabilistic graphic model (PGM) to
solve classification problem on elementary school students hande writing. Specifically,
I will first train a PGM model and then use it to infer different grade student's hande writing
among 1st to 4st grade student. During the implementation, I will also calcuate the entropy,
KL-divergence based on direct computation or approximate inference.
In the final step, I will compare these two approaches and give a detailed illustration.
\subsection{Dataset Description}
We have two major data sets. First is cursive writing and the other is printed writing.
Printed writing data are extracted from 1st-4st grade students and cursive writing data 
are extracted from 3rd-4st grade students. To simplify the analysis, we only make classification
on word \textbf{"and"}, that is all data set only contain infomation with "and" word.
In order to clarify and help understanding, I post each features meaning and value, which are shown
below.
\begin{itemize}
\item Initial stroke of "a"
\item Formation of "a" staff
\item Number of "n"arches" 
\item Shape of "n" arches
\item Location of "n"mid
\item Formation of "d" staff
\item Formation of "d" initial
\item Formation of "d" terminal
\item Symbol
\item a-n relationship
\item a-d relationship
\item n-d relationship
\end{itemize}

The total features of printed are shown above. Each feature may contain three or
four values. However, in pratice, it is possible that some features may have
invalid value, like "-1" for null data or "99" for inconsistent data. The detail
of how to handle these kinds of value will be discussed in next section
\subsubsection{Handle Invalid Data}
Handling invalid or missing data is one of the major steps in data preprocessing
step. The handwriting data set of our project contain large amount of missing data
"-1" and inconsistent data "99". If just ignoring them, it will obviously harm our
model. So, an effective method must be put forward to solve this.
My method of handling invalid data is single imputation, specifically using mode
substituion. By calcuationg the mode value of each feature in each class, I simply
replace the original invalid data with new data, the mode value.

\section{Inference Algorithms} 
In this part, I am going to use two approaches to make inference. The first is to 
directly computute the value from data set. Another approach is to use Approximate
Inference called Monte Carlo method. In this method, samples are produced by sampling
from PGM models using common sampling approach like ancestral sampling or Gibbs sampling.
\subsection{Mean of Distribution}
First, I will use data sets to calcuate the mean feature value of each grade.
The formula is shown below.
\begin{equation}
	\label{eq:mean-hat}
	\hat{E}[p(x)]=\frac{1}{N}\sum_{N}^{i}x
\end{equation}

The result of direction computation is shown below.
\begin{figure}[ht]
		\vskip 0.1in
		\begin{center}
				\centerline{\includegraphics[width=\columnwidth]{mean_printed}}
				\caption{mean value of printed writing, using data from all grades}
		\end{center}
		\vskip -0.1in
\end{figure}

\subsection{Most similar samples}
The formula of calcuating the most similar samples to the mean is
shown below.
\begin{equation}
		d(X,Y)=\frac{1}{D}\sum_{D}^{i=1}\frac{|X-Y|}{max(X)}
\end{equation}
Since some features' maximun value is 0, we must manually add one
additional sample which all columns are 1 to prevent dividing by zero.
We can easily know this won't hurt the data set so much.
The distance of the most similar 4 samples to the mean of each grade is show below.
\begin{figure}[ht]
		\vskip 0.1in
		\begin{center}
				\centerline{\includegraphics[width=\columnwidth]{dist_most_sim_value_printed}}
				\caption{The distance of the most similar 4 samples to the mean of each grade}
		\end{center}
		\vskip -0.1in
\end{figure}

\subsection{Entropy of Distribution}
To calcuate the entropy, we first count the number of each value of each feature.
After that, we use the entropy formula equation shown below to calcuate it.
\begin{equation}
	\label{entropy}
	E[p(x)]=-\sum_{x}p(x)\ln p(x)
\end{equation}
Also in order to prevent zero probability, we need to manually add
several samples depending on the values each feature can take.
The entropy of each grade is shown below.

\begin{figure}[ht]
		\vskip 0.1in
		\begin{center}
				\includegraphics[width=\columnwidth]{entropy_printed}
				\caption{entropy of each grade}
		\end{center}
		\vskip -0.1in
\end{figure}

\subsection{Relative Entropy between two Distributions}
Since there are 4 classes of printed writing and 2 classes of cursive writing, we
need to calculate x relative entropy values using the formula shown below.
\begin{equation}
	KL(p||q)=-\sum_{x}p(x)[\ln q(x) - \ln p(x)]
\end{equation}

The result is shown below.
\begin{figure}[ht]
		\vskip 0.1in
		\begin{center}
				\includegraphics[width=\columnwidth]{relative_entropy_printed}
				\caption{relative entropy between each grades}
		\end{center}
		\vskip -0.1in
\end{figure}

\subsection{Unusual writing}
To get the most unusual writing, I use likelihood function and calculate negative log value
of each sample. The most unusual writings of each grade and each writing type are listed below.
\begin{equation}
		value=-\ln {\prod p(x)}
		=-\sum_{x}\ln {p(x)}
\end{equation}

\begin{table}[t]
\vskip 0.15in
\begin{center}
		\begin{tabular}{ | l | l | l | p{5cm} |}
				\hline
				Grade printed & Numbers & Value \\ \hline
				grade 1 	& 5  & 3.8755    \\ \hline
				grade 2 	& 71 & 2.8934    \\ \hline
				grade 3 	& 61 & 2.5030    \\ \hline
				grade 4 	& 49 & 2.7162  \\
				\hline
		\end{tabular}
\end{center}
\vskip -0.5in
\begin{center}
		\begin{tabular}{ | l | l | l | p{5cm} |}
				\hline
				Grade cursive & Numbers & Value \\ \hline
				grade 3 	& 25 & 2.8662    \\ \hline
				grade 4 	& 18 & 3.4497 	 \\
				\hline
		\end{tabular}
\end{center}
\caption{Unusual writing of each grade and each writing type}
\vskip -0.1in
\end{table}
\subsection{Results and Discussion}
In direct computation step, we calculate the mean, entropy, relative entropy of each grade.
Based on the results, we can easily find that relative entropy between each grades gives
the most significant explanation. According to Figure 4, we can give such a description.
KL divergence between grade 1 and 2, grade 2 and 3 or grade 3 and 4 are small, while this
value is much larger between grade 1 and 3 or grade 1 and 4 or 2 and 4. This is due to children's
handwriting keeping on changing from grade 1 to 4. So, the KL divergence, which measures the extent
of difference, is larger between 1 and 3 than 2 and 3.
\section{Learning Bayesian Network}
In this section, A Probablisitic Graphic Model of Bayesian network will be built up and I will use it to do approximate
sampling. The main process will be first learning the graph structure and obtaining the conditional probability, after
that, we can use this model to do sampling like gibbs sampling.
\subsection{Learning Graph Structrue}
The first step of using Bayesian network is to construct its graph structure.
There are two major structure learning algorithms. First is called constrained-based
algorithm. Another one is called search and scoring alogrithm. Since we need
to test probabilistic dependence relationship. I will use this algorithm to construct the
graph structure.
In order to reduce the space of possible DAGs i, we need to add some constrains. The method in
this article mainly concerned two kinds of constrains. First is constraints on parameters and 
the second is structure constraints.
For both constraints, we have following assumptions.

\begin{itemize}
		\item The variables in the database (sample) are discrete. 
		\item The cases occur independently given the underlying probabilistic model of the data.
		\item The volume of the data is large enough for the reliable independence tests used in the algorithm proposed here. 

\end{itemize}
Besides assumpions mentioned above, we also have several additional independence assumptions.
We assume the data come from a mutinominal distribtuion. Since mutual information 
can be represented below
\begin{equation}
		H(X:Y)=I(X,Y)=\sum_{x,y}P(x,y)\ln \frac{P(x,y)}{P(x)P(y)}
\end{equation}
and we know a measure that can be used to establish the dependent and independent relationship
between two node, which is shown below.
\begin{equation}
		T = 2 N \hat I
\end{equation}
where N is the size of the sample and $\hat I$ can be the equation (eq.6).
Under independence assumption and under the hypothesis that the data come from
a multinomial distribution, the value of T is aysmptotically distributed as a 
$\chi^2$ variable with (X-1)(Y-1) degrees of freedom. From this knowledge,
we can then perform an independence statistical test to check whether two variables
in a Bayesian network are marginally or conditionally dependent or independent.
\subsection{Algorithm Description}
1. Compute the value of the mutual information that each variable
Yi ($ 1 \leq i \leq n $) provides to X and order them according the information
they provide from the largest to the smallest value. The order variables
will be labelled now Y(1), Y(2),...,Y(n). Draw the arc from Y(1) to X. and check whether
the hyothesis that two variables are independent from each other holds or not.
If H0 does not hold then draw a directed arc from Y(i) ($ 2 \leq i  \leq n$ ) to X.
\\2. Compute the value of the conditional mutual information that each
Y(i) provides to X given Y(1).
\\3. Do X=Y(1) and delete Y(1) from the set of independent random variables
Let the number of independent variable n=n-1. If $n \geq 2$ then goto step 1; otherwise
stop.
\subsection{structure description}
The final result of each grade's graph is show below.
\begin{figure}[h]
		\vskip 0.1in
		\centering
				\centerline{\includegraphics[width=.7\columnwidth]{graph_grade_1_printed}}
				\caption{The graph generated using grade one printed data set}
		\vskip -0.1in
\end{figure}
From the figure 5, we know that v0 depends on
v1. v3 dependes on v2 and v6 depends on v7.
\begin{figure}[h]
		\centering
				\centerline{\includegraphics[width=.9\columnwidth]{graph_grade_2_printed}}
				\caption{The graph generated using grade two printed data set}
\end{figure}
From figure 6, we know that v5, v7 has common parent
node v6. v9 and v10 also has common parent v11. We can
see this graph is a little complicated than graph from
grade one.
\begin{figure}[h]
		\begin{center}
				\centerline{\includegraphics[width=.5\columnwidth]{graph_grade_3_printed}}
				\caption{The graph generated using grade three printed data set}
		\end{center}
\end{figure}
\begin{figure}[h]
		\begin{center}
				\centerline{\includegraphics[width=.5\columnwidth]{graph_grade_4_printed}}
				\caption{The graph generated using grade four printed data set}
		\end{center}
\end{figure}

\begin{table}[h]
%\vskip 0.1in
\begin{center}
		\begin{tabular}{ | l | l | l | p{5cm} |}
				\hline
				Grade 		 	& iterations   \\ \hline
				grade 1 		& 19      \\ \hline
				grade 2 		& 652     \\ \hline
				grade 3 		& 95     \\ \hline
				grade 4 		& 101   \\
				\hline
		\end{tabular}
\end{center}
\caption{iterations times used by each grade data to train the graph structure}
%\vskip -0.1in
\end{table}

\subsection{Inference}
\subsubsection{mean calculation}
Using the sampling data to calcuate the mean value, the 
result is shown in figure 9.
\begin{figure}[h]
		%\vskip 0.1in
		\begin{center}
				\centerline{\includegraphics[width=\columnwidth]{s_mean}}
				\caption{the mean value of using sampling data}
		\end{center}
		%\vskip -0.1in
\end{figure}
\subsubsection{entropy}
Using the sampling data to calculate entropy, the result
is shown in figure 10.
\begin{figure}[h]
		\vskip 0.1in
		\begin{center}
				\centerline{\includegraphics[width=\columnwidth]{s_entropy}}
				\caption{the entropy value of using sampling data}
		\end{center}
		\vskip -0.1in
\end{figure}

\subsubsection{KL divergence}
Using the sampling data to calculate KL divergence, the result
is shown in figure 11.
\begin{figure}[h]
		\vskip 0.1in
		\begin{center}
				\centerline{\includegraphics[width=\columnwidth]{s_kl}}
				\caption{the KL divergencevalue of using sampling data}
		\end{center}
		\vskip -0.1in
\end{figure}

\section{Conclusion} 
From the work mentioned above, we can see using probabilistic graphic model
to make infernece is effective, in spite of model complexcity to some extent.
Given the test result, we can know that children's handwriting keep on changing
and thus the value of relative entropy between grade 1 and 3 is larger than
2 and 3.
\\Due to limited amount of preparation,  a lot of work still need to
be improved, such as algorithm efficency and space efficiency analysis. This
part of work will be included in the next report.

%\begin{algorithm}[tb]
%   \caption{Bubble Sort}
%   \label{alg:example}
%\begin{algorithmic}
%   \STATE {\bfseries Input:} data $x_i$, size $m$
%   \REPEAT
%   \STATE Initialize $noChange = true$.
%   \FOR{$i=1$ {\bfseries to} $m-1$}
%   \IF{$x_i > x_{i+1}$} 
%   \STATE Swap $x_i$ and $x_{i+1}$
%   \STATE $noChange = false$
%   \ENDIF
%   \ENDFOR
%   \UNTIL{$noChange$ is $true$}
%\end{algorithmic}
%\end{algorithm}
 
 
% Note use of \abovespace and \belowspace to get reasonable spacing 
% above and below tabular lines. 

%\begin{table}[t]
%\caption{Classification accuracies for naive Bayes and flexible 
%Bayes on various data sets.}
%\label{sample-table}
%\vskip 0.15in
%\begin{center}
%\begin{small}
%\begin{sc}
%\begin{tabular}{lcccr}
%\hline
%\abovespace\belowspace
%Data set & Naive & Flexible & Better? \\
%\hline
%\abovespace
%Breast    & 95.9$\pm$ 0.2& 96.7$\pm$ 0.2& $\surd$ \\
%Cleveland & 83.3$\pm$ 0.6& 80.0$\pm$ 0.6& $\times$\\
%Glass2    & 61.9$\pm$ 1.4& 83.8$\pm$ 0.7& $\surd$ \\
%Credit    & 74.8$\pm$ 0.5& 78.3$\pm$ 0.6&         \\
%Horse     & 73.3$\pm$ 0.9& 69.7$\pm$ 1.0& $\times$\\
%Meta      & 67.1$\pm$ 0.6& 76.5$\pm$ 0.5& $\surd$ \\
%Pima      & 75.1$\pm$ 0.6& 73.9$\pm$ 0.5&         \\
%\belowspace
%Vehicle   & 44.9$\pm$ 0.6& 61.5$\pm$ 0.4& $\surd$ \\
%\hline
%\end{tabular}
%\end{sc}
%\end{small}
%\end{center}
%\vskip -0.1in
%\end{table}

\nocite{langley00}


\end{document} 
